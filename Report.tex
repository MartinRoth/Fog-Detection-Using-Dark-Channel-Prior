\documentclass{article}

\title{Fog detection from camera images}
\begin{document}

\begin{abstract}

Fog is one of the most dangerous weather types with more fatalities than winter storms. It is in the interest of the general public that a precise, predictive and accurate fog density map with high spatial resolution can be created. Currently the definition of fog as used by national weather services can only be applied at a few locations by means of light scattering experiments. With the rising availability of cameras in public places such as airports, streets and highways a large amount of data on the occurrence of fog becomes available to researchers. In this article we describe a method for determining not only the existence of fog, but the visibility distance - a type of optical penetration length - as well. We will show that video cameras are a reliable alternative or complementary method for creating fog visibility maps when the dark channel prior method is used. Not only detection but also visibility ranging correlated to the current meteorological standard for visibility ranging can be done with video camera images. 
\end{abstract}

\section{Introduction}

Fog is the weather phenomenon of light scattering particles - usually water droplets - suspended in air causing an attenuation of light and therefore a severely reduced visibility of objects. The sudden appearance of fog - especially dense fog - can lead to such a reduced visibility that transportation networks can be affected or even fully compromised: for example massive car collisions resulting in long traffic jams, grounding of airplanes or even closing of airports and reduced speed of trains to prevent derailment. Some of these effects can be alleviated or even prevented when a transportation network can adjust to a high spatial resolution accurate fog density map by issuing warnings or decreasing the speed limit. Unfortunately such a density map needs a dense network of sensors capable of detecting fog and measuring the visibility distance, a network weather services are now lacking. Current fog detection systems measure the amount of scattering of a collimated beam of infrared light to determine the Meteorological Optical Range  (MOR): the distance at which a collimated beam of incandescent light with a light colour of 2700K has reduced to an amount of 5% of the emitted flux.
In the Netherlands there are about 15 sites capable of determining the MOR resulting in a spatial resolution of 50 km. This spatial resolution is much larger than typical length scales on which fog changes significantly. Therefore a new and complementary method based on new data sources is needed. The rising spread of public cameras for control, security and safety allows for a much denser network of fog detecting sensors. For example The Netherlands has about 7000 traffic cams, which would yield a spatial resolution of about 1 km if the cameras are distributed evenly over the nation. Unfortunately the meteorological definition of fog is not only based on the detection/measurement of light intensity but on the emitted light intensity as well. Cameras can only do the detection/measurement of light intensity. Thus the current definition of visibility is not suitable for cameras. Therefore different properties of fog have to be used in a camera based fog detection system. For validity this fog system must correlate to the MOR and fog detection and classification based on human perception. 

\section{Properties of fog and their measurement}

As we stated before fog is the weather phenomenon of light scattering particles, suspended in air causing an attenuation of light and therefore a severely reduced visibility of objects. This description already hints to several characteristic properties. The most important parts of this description are "light scattering particles" and "reduced visibility of objects". The first part implies that light of a source can be seen from a direction different than the source direction. As a result the total amount of light scattered into one specific direction will lead to a shift of an object colour towards white. Furthermore the attenuation due to the scattering leads to a gradual change of the fog colour from white to black depending on the attenuation length of the fog and the thickness of the fog layer.
The second part of the description indicates a loss of resolution and shows that visibility is a relative quantity depending on the no fog perception of an object. In general these properties can be interpreted as combinations of smearing and averaging effects. 
The MOR detection method is fully based on scattering but does not include loss of resolution or the change of colour. Loss of resolution can be quantified by edge detection via gradient thresholding or high level wavelet transforms or by spectral analysis via high frequency Fourier transform. The colour shift can be quantified by comparison between the RGB-channels of the cameras.

\section{ Ranging}

Fog detection systems should not only determine the existence of fog but classifying fog as well. A crude method will distinguish between no fog, mild fog and dense fog, but a sophisticated method can actually do the ranging similarly to the MOR method. One can do the ranging in two ways: statistical correlation with MOR data or assigning geometric distances to fog threshold level locations in pictures. 
The correlation method needs a relatively large set of properties in order to have a reasonable chance  to find a correlated property with respect to the MOR data. The downside of this method is the possibility that the found relation is a statistical anomaly. 
The geometric method already works when a fog threshold level has been found in a picture. However the validity of this method has to be checked by comparison with the MOR data.
We have chosen to do the ............... method. 

......... Method

\end{document}
